\section{Related Work}
\label{sec:RelatedWork}
\subsection{Shifter}
Shifter \cite{jacobsen2015contain} is an execution environment also aims to provide containerized environment for HPC systems. Since deploying standard Docker daemon on HPC systems imposes security and compatibility issue, they build a Docker-like container environment, which provides portability, isolation and reproducibility like Docker. The Shifter runs customized Shifter image. Docker users need to first import their Docker images and convert them into Shifter images before running. Shifter container can access host file system via volume mapping, however there is no explicit application data management. Also, sharing files between containers requires the file sharing abilities between host machines. Shifter can only be deployed on customized Cray machines with root privileges. On the other hand, BEE can run standard Docker images unmodified, which provides a higher usability for users. For example, developers can easily deploy consistent environment across their local development and test machines with production environment. Using standard Docker brings more convenience to share Docker image in Docker community. BEE has explicit application data management than can enable easy transfer across host machines for live migration or work flow integration. Also, BEE has several modes for data file sharing between processes that can be configured by user depends on whether file sharing is enabled on the hosts. If not, BEE can build its own file sharing mechanism, which brings more flexibility. Finally, BEE can be deployed on any HPC system and even cloud system without root privileges. 
\subsection{Singularity}
Singularity \cite{kurtzer_2016_60736} is another containerized execution environment for HPC systems. Similar to Shifter it also build a Docker-like container execution environment to run customized Singularity images. Standard Docker images are supported but they also need to be converted to Singularity image before running. It is also required to have root privileges in order to deploy Singularity on a HPC system. But, unlike Shifter, it can be deployed on any HPC systems. It doesn't have a management on application data. Data sharing between containers also depends on the host machines. With multiple configuration solutions, BEE has more flexibility on deploying on HPC system. Besides providing an containerized execution environment, BEE bring better usability by combining data management, workflow integration, live migration, and cloud computing together to provide more convenient tool for HPC users and developers.
\subsection{Charliecloud}
Charliecloud \cite{priedhorsky2016charliecloud} is a container solution that brings Docker-like environment into HPC system. It brings all the benefits of standard Docker container. The main benefit of this it that user or developers can have consistent building and execution environment across their local development or test machines to large scale production cluster machines. However, installing Charliecloud environment requires that the target HPC system has Linux kernel version to be least 3.18. It is challenging to install Charlieclound on current HPC system. It would take years before mainstream HPC system can upgrade Linux kernel that can meet the requirement of Charliecould. Also, it only manages the execution environment, however many other aspects of overall user workflow have not been considered.
\subsection{AWS Container}
Amazon EC2 Container Service(ECS) \cite{awscontainer} is a Docker container service provided by Amazon on AWS. Since ECS deploys on top of EC2, and EC2 has a layer on virtual machines, ECS actually deploy Docker container layer on top of virtual machine layer, so basically it has the similar host-vm-docker structure as BEE. Regardless of the underlying hardware configuration, ECS provide a consistent building and execution environment by using standard BEE. Docker users can easily run their application on ECS without modification. That's why we deploy the similar structure on both cloud and HPC environment in BEE. 
  

